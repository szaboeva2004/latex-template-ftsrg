\chapter{BTOR2 Frontend in Theta}

Megemlít BTOR2C
Theta sok sok frontend ( ez inkább background )
Közvetlenül BTOR2 leképezés mert mi dönthetjük mit akarunk csinálni, optimizálni, remény: Theta teljesítménye jobb lesz mint köztes nyelv pl C verifikálva
Better performance, more precise analysis (ez így igazából a motivation)
\section{Motivation}


Kéne rajzolni egy általános CFA-t hogy hpgy képeződik le kódból. Draw io ban megrajzolni

\section{BTOR2 to CFA transformation}
(NAGGGGYOn nagy rész) miből lesz változó, mi hova kerül
Rajz rajz yap yap magyaráz yap

inites, bades, nextes, mindenes
Btor2 különböző nodejai hova kerülnek
Táblázatban vagy listában ahogy jól esik
CFA figureon minden doboz egy subsectionben van elmagyarázva

Section/subsection count2 btor2 kódja és mellérakjuk a CFAt de csak én rajzolgatva, aztán erről yapyapyap

\section{Implementáció}
Hogyan és mi kerül edgebe (végülis ez a trafónál már kiderül) 
Visitor minta, ANTLR, unexpected challanges minusztt kellett implementálni, nincsen benne a dokumentációjukban hogy ezt szaba :'Ccc 

\section{Current limitations}
Vannak bennen olyan operátorok, amik a Thetában nincsenek, Overflow predikátumok, Arrayeket nem támogatjuk -> Future work hihihihih (sosem szabadulunk) 
Csak a bad property van (safety) Something bad should never happen
Experimental feature thetában a liveness : Something good should always happen leképezésmég bonyibb lesz so az nincsen
 
\section{Prelimanary results}
Minden szép és jó és benchmark és lesz ez még jobb is és gyakorlás és yipi és legyen meg
Kicsike kis táblázatocska

\section{Conclusion}
Összefoglalás
Mit csináltam
mi sikerült

\subsection{Future work}
Overflow operators, kimaradt egy két dolgo, liveness in Theta, Nagyobb pontok: witness generation (kell a hwmcc részvételhez)
Nagyobb kiértékelés, sok debug, lesz ez még jobb is, optimalizációk, CFAn optimalizációs passok

Ha LateX csúnya: szólni Zsófinak

Itt van, hogy én mit csináltam: 
