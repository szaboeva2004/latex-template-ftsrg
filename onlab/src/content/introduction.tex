%----------------------------------------------------------------------------
\chapter{\bevezetes}
%----------------------------------------------------------------------------

In recent years, the field of formal verification has gained increasing importance in both academic research and industrial practice, as the correctness and reliability of software and hardware systems have become critical concerns. Model checking, one of the central techniques in this domain, provides an automated way to rigorously verify system properties by exhaustively exploring all possible behaviors of a model.~\cite{modelChecking}\cite{systemEngineering}

\paragraph{Motivation} My project was inspired by a concrete and practical challenge: the lack of a dedicated BTOR2 frontend in the Theta model checking framework. Theta is a flexible and modular verification tool developed at the Budapest University of Technology and Economics, capable of analyzing a wide range of model types using advanced abstraction-refinement techniques. While Theta supports various frontends for higher-level formalisms such as C and statecharts, BTOR2 - a popular word-level model checking format widely used in hardware verification - has not yet had an officially supported frontend. Bridging this gap motivated my work.~\cite{btor2}\cite{theta}

BTOR2, developed in the context of the Hardware Model Checking Competition (HWMCC)\footnote{\url{https://hwmcc.github.io/}}, is a sorted, line-based intermediate format tailored to accurately model bit-precise sequential systems. It supports complex operations over bit-vectors and arrays while maintaining a structure that is both compact and analyzable. Existing tools such as Btor2C can translate BTOR2 into C, but they are not designed for seamless integration with model checkers like Theta, nor do they offer a flexible way to perform model transformations or leverage Theta's advanced verification features.~\cite{btor2c}

\paragraph{Solution} To address this problem, I implemented a custom BTOR2 grammar using ANTLR4, a powerful parser generator framework. ANTLR allows for the generation of parsers in multiple target languages while providing a clean separation between syntax and semantics through visitor and listener patterns. This made it an ideal choice for processing BTOR2's rigid line-based syntax and constructing intermediate representations such as Control-Flow Automata (CFA), which serve as the internal model format in Theta.\cite{cfa}

My contribution consists of a BTOR2-to-XCFA parser pipeline integrated with Theta. The parser reads BTOR2 files, constructs an XCFA, enabling the use of Theta's reachability analysis to verify hardware-like systems described in BTOR2. In my implementation, I focused on handling the bit-vector operations and modeling simple sequential systems, such as counters or state machines. The toolchain makes it possible to verify these models using the same formal reasoning techniques previously applied only to software.
