\chapter{Conclusion}\label{chapter:conclusions}

This thesis detailed the design, implementation, and evaluation of a direct \textsc{Btor2}-to-CFA transformation within the Theta model checking framework. The work was motivated by the inefficiencies and potential semantic discrepancies introduced by existing translation-based workflows that convert hardware models to C code before verification.

The implemented BTOR2XCFA frontend successfully demonstrates that a direct approach is not only feasible but also structurally and algorithmically superior. The evaluation on a comprehensive suite of HWMCC benchmarks across 8,316 runs revealed that:

\begin{enumerate}
    \item The direct transformation generates models that are significantly smaller and less complex than those produced by the Btor2C-based pipeline (\textbf{8.87$\times$ smaller files}, \textbf{3.46$\times$ lower complexity}).
    \item This structural efficiency directly translates to better verification performance, with \textsc{Btor2} inputs enabling \textbf{2.4$\times$ higher algorithm performance scores} and \textbf{72\% faster verification times} for the best-performing algorithm (CEGAR\_EXPL).
    \item The approach maintains precise bit-vector semantics while naturally representing hardware cycles as graph loops, avoiding the semantic gaps of translation-based methods.
\end{enumerate}

The reduction in complexity -- evident in file size, label count, control-flow structure, and algorithm performance -- strongly demonstrates that the direct approach leads to more efficient model checking in terms of both memory usage and processing time.

In conclusion, this work provides compelling evidence for the advantages of direct, semantics-preserving transformations in formal verification toolchains. By integrating hardware verification directly into a powerful, extensible framework like Theta, we can leverage advanced software verification algorithms without sacrificing the performance and precision required for hardware analysis.

\section{Future Work}

The development of the \textsc{Btor2} frontend opens several avenues for future work to enhance its capabilities, robustness, and performance.

\begin{enumerate}
    \item \textbf{Language Completeness:} The highest priority is to achieve full \textsc{Btor2} compliance. This involves:
    \begin{itemize}
        \item Implementing support for unsupported operators, such as overflow-detecting arithmetic predicates (saddo, uaddo, etc.).
        \item Adding support for array operations, which is crucial for modeling memory and other complex hardware components.
    \end{itemize}
    
    \item \textbf{Extended Property Verification:} Implementing the translation of \textsc{Btor2} liveness properties into Theta's formalism would significantly expand the frontend's applicability, allowing it to verify a broader class of hardware specifications.
    
    \item \textbf{Algorithm Optimization:} Based on the performance results, further optimization of CEGAR\_EXPL and CEGAR\_PRED algorithms specifically for \textsc{Btor2} inputs could yield additional performance improvements.
    
    \item \textbf{Witness Generation:} To become competitive in formal verification competitions like HWMCC, the tool must be able to generate witnesses (counterexamples or proofs). Integrating witness generation for both valid and invalid properties is an essential next step.
    
    \item \textbf{Performance Optimization:} Developing \textsc{Btor2}-specific optimization passes on the generated XCFA can further improve performance and scalability. These could include simplifications based on hardware circuit patterns that are not apparent in general-purpose software.
    
    \item \textbf{Comprehensive Benchmarking and Debugging:} As the frontend matures, more extensive benchmarking on a wider array of circuits, including those using newly implemented features, is necessary to evaluate its robustness, correctness, and performance relative to state-of-the-art hardware verifiers like ABC and AVR.
\end{enumerate}

With these improvements, the \textsc{Btor2} frontend for Theta has the potential to evolve into a competitive, comprehensive, and high-performance solution for hardware model checking.

\begin{table}[h!]
\centering
\caption{Conceptual comparison of BTOR2 (Simple) and C (Complex) frontends}
\label{tab:btor2_vs_c_conceptual}
\renewcommand{\arraystretch}{1.2}
\begin{tabular}{@{}p{4cm}p{5cm}p{5cm}@{}}
\toprule
\textbf{Category} & \textbf{Btor2CFA} & \textbf{C2CFA} \\
\midrule
Variable dependencies & 
Straightforward variable relations & 
Many extra variables created when calling functions \\

Temporary variables & 
No extra temporary variables & 
Lots of temporary variables needed for copying values and parameters \\

State handling & 
State updated directly & 
State changed indirectly through function calls and parameters \\

Bit-level operations & 
Bit-sizes handled automatically & 
Bit-sizes handled manually using masks \\

Procedures and structure & 
One simple procedure & 
Several procedures, causing call overhead \\

Instrumentation and semantics & 
Very little additional logic & 
More complex logic due to C semantics and assertion checks \\
\bottomrule
\end{tabular}
\end{table}




\begin{table}[h!]
\centering
\caption{Example XCFA statistics for BTOR2 and C frontends}
\label{tab:btor2_vs_c_xcfa}
\renewcommand{\arraystretch}{1.2}
\begin{tabular}{@{}p{4.5cm}p{5cm}p{5cm}@{}}
\toprule
\textbf{Aspect} & \textbf{Btor2CFA} & \textbf{C2CFA} \\
\midrule
Procedures & 1 & 3 \\
Variables & 6 & 16 \\
Locations & 7 & 21 \\
Edges & 7 & 20 \\
Bit-width & 3-bit counter with explicit loop & 8-bit variables with explicit masking \\
Architecture & Compact, direct state machine & Complex, C-semantics-based model \\
Instrumentation & Minimal overhead & External assertion checking infrastructure \\
\bottomrule
\end{tabular}
\end{table}
