\chapter{Conclusion}\label{chapter:conclusions}

This thesis detailed the design, implementation, and evaluation of a direct \textsc{Btor2}-to-CFA transformation within the Theta model checking framework. The work was motivated by the inefficiencies and potential semantic discrepancies introduced by existing translation-based workflows that convert hardware models to C code before verification.

The implemented BTOR2CFA frontend successfully demonstrates that a direct approach is not only feasible but also structurally and algorithmically superior. The evaluation conclusively shows that:

\begin{enumerate}
    \item The direct transformation produces significantly more compact CFAs, reducing variable counts by two-thirds and statement counts by over 80\%.
    \item Verification performance is dramatically improved, with higher success rates, lower memory usage, and faster execution times across all tested algorithms.
    \item Optimization passes are essential for scalability, enabling the verification of complex circuits that would otherwise be intractable.
    \item The direct approach better preserves hardware semantics, enabling more accurate and efficient verification compared to indirect translation via C.
\end{enumerate}

These findings validate the hypothesis that eliminating intermediate translation steps in hardware verification workflows can lead to substantial gains in both performance and semantic precision.

\section{Future Work}

The development of the \textsc{Btor2} frontend opens several avenues for future work to enhance its capabilities, robustness, and performance.

\begin{enumerate}
    \item \textbf{Language Completeness:} The highest priority is to achieve full \textsc{Btor2} compliance. This involves:
    \begin{itemize}
        \item Implementing support for unsupported operators, such as overflow-detecting arithmetic predicates (saddo, uaddo, etc.).
        \item Adding support for array operations, which is crucial for modeling memory and other complex hardware components.
    \end{itemize}
    
    \item \textbf{Extended Property Verification:} Implementing the translation of \textsc{Btor2} liveness properties into Theta's formalism would significantly expand the frontend's applicability, allowing it to verify a broader class of hardware specifications.
    
    \item \textbf{Algorithm Optimization:} Based on the performance results, further optimization of CEGAR\_EXPL and CEGAR\_PRED algorithms specifically for \textsc{Btor2} inputs could yield additional performance improvements.
    
    \item \textbf{Witness Generation:} To become competitive in formal verification competitions like HWMCC, the tool must be able to generate witnesses (counterexamples or proofs). Integrating witness generation for both valid and invalid properties is an essential next step.
    
    \item \textbf{Comprehensive Benchmarking and Debugging:} As the frontend matures, more extensive benchmarking on a wider array of circuits, including those using newly implemented features, is necessary to evaluate its robustness, correctness, and performance relative to state-of-the-art hardware verifiers like ABC and AVR.
\end{enumerate}

With these improvements, the \textsc{Btor2} frontend for \textsc{Theta} has the potential to evolve into a competitive, comprehensive, and high-performance solution for hardware model checking.