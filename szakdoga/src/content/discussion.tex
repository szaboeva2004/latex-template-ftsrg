\chapter{Discussion}\label{chapter:discussion}



\section{Implementation Considerations and Optimization Trade-offs}

\subsection{Theta's C Frontend: A Mature but Complex Infrastructure}

Theta's existing C frontend is a mature component capable of handling a wide range of software verification tasks. However, its generality comes at a cost:

\begin{itemize}
    \item \textbf{Structural Overhead:} The C frontend introduces auxiliary variables and explicit memory operations to model software semantics, leading to bloated CFA representations.
    \item \textbf{Semantic Mismatch:} Hardware constructs such as arbitrary-width bit-vectors and cyclic state updates must be emulated through C idioms, which can obscure the original circuit behavior and hinder optimization.
\end{itemize}

\subsection{BTOR2 Frontend: Minimalist and Domain-Specific}

In contrast, the \textsc{Btor2} frontend is designed with hardware semantics in mind:

\begin{itemize}
    \item \textbf{Direct Mapping:} Bit-vector types map directly to SMT solver types, preserving precise semantics without intermediate representation.
    \item \textbf{Graph-Based Loops:} Cyclic behavior is naturally expressed as CFA loops, avoiding the need for explicit unrolling or simulation constructs.
    \item \textbf{Targeted Optimizations:} Domain-specific optimizations such as Large-Block Encoding and variable elimination are applied directly to the CFA, reducing redundancy and improving solver performance.
\end{itemize}

\section{Limitations and Their Impact}

The current implementation has several limitations that affect its applicability and performance:

\begin{itemize}
    \item \textbf{Incomplete Language Support:} Missing operators (e.g., \texttt{saddo}, \texttt{uaddo}, array operations etc.) restrict the range of verifiable circuits. This limits the frontend's utility in industrial settings where such features are common.
    \item \textbf{No Witness Generation:} The inability to produce counterexamples or proofs hinders integration into formal verification competitions and industrial workflows that require actionable feedback.
    \item \textbf{Limited Property Types:} Support is currently limited to safety properties. Liveness and other temporal properties remain unhandled, narrowing the scope of verifiable specifications.
\end{itemize}

Despite these limitations, the BTOR2CFA frontend demonstrates clear advantages in efficiency and semantic fidelity for supported circuits. Future work addressing these gaps will further enhance its competitiveness and utility.