\chapter{The transformation from BTOR2 to (X)CFAs \todo{Or just simply Implementation?}}

\section{Overview}
\todo{Csak hogy ne legyen elfelejtve}

\section{Motivation}\label{sec:motivation}

\section{Theta}\label{sec:theta}

Regarding the implementation, I chose Theta~\cite{theta}, a modular and extensible model checking framework developed by the Critical Systems Research Group at the Budapest University of Technology and Economics, as the target platform for integrating a BTOR2 frontend.

The main advantage of Theta is its support for the combination of various abstract domains, interpreters, and strategies for abstraction and refinement, applied to models of various formalisms with higher level language front-ends (e.g. for C programs, or in my case BTOR2 circuits).

To begin, I defined an unofficial BTOR2 grammar using ANTLR4~\cite{antlr}, a powerful parser generator widely adopted in both academia and industry for constructing interpreters, compilers, and domain-specific languages.

Theta supports abstraction-refinement based reachability analysis. In my project, I worked with XCFA models—an extension of Control Flow Automata (CFA) that incorporates procedures and concurrency - but focused primarily on the CFA subset. This allowed me to apply Theta's robust CEGAR-based algorithms for reachability checking.



\section{BTOR2 frontend in Theta}
\todo{Yap what is Theta and how wondeful the many fronetnds are}


\subsection{Walkthrough of transformation to CFA}
% What elemnts does BTOR2 have
% Which locations are what
% What are on edges
% reachability testing
% Bad properties

\subsubsection{Example BTOR2 to CFA}

\section{C frontend in Theta}

\todo{Comparison between a c2CFA and BTOR2XCFA}