\chapter{Related work}

% Btor2C
% Btor2Cert
% State Space-Based Methods for Validating Model Transformations in Model Checkers
\section{Related work}

\subsection{Btor2C and Translation-Based Workflows}

Btor2C [5] represents the state-of-the-art in hardware-to-software translation for verification purposes. This tool systematically converts BTOR2 circuits into C programs, enabling software verification tools to analyze hardware designs indirectly. The translation workflow allows leverage of sophisticated software verification techniques—such as those in CPAChecker [3] and ESBMC [16]—on hardware circuits. Empirical results have demonstrated that this approach can uncover bugs that dedicated hardware verifiers like ABC [9] and AVR [17] miss under identical time constraints.

However, the Btor2C approach introduces a multi-step translation process: BTOR2 → C → CFA → verification. Each transformation step potentially introduces semantic discrepancies and optimization barriers. Furthermore, the translation must bridge fundamental differences between hardware and software paradigms, particularly regarding arbitrary-width bit-vector types and cyclic behavior representation.

\subsection{Comparative Analysis of Verification Workflows}

Existing verification workflows for hardware systems can be broadly categorized into two principal approaches, each with distinct characteristics and trade-offs.

\subsubsection{Direct Hardware Verification}

The direct hardware verification approach employs specialized tools such as \textbf{ABC} and \textbf{AVR} that operate directly on hardware description formats. These tools follow a streamlined verification path: \(\text{BTOR2} \rightarrow \text{direct verification}\). The primary advantage of this methodology lies in its preservation of native hardware semantics and its optimization for circuit-specific structures, enabling highly efficient analysis tailored to digital logic. However, this approach faces limitations in its ability to incorporate recent advances developed within the software verification community, potentially missing out on innovative analysis techniques that have emerged from that domain.

\subsubsection{Translation-Based Verification}

Translation-based verification represents an alternative paradigm that bridges hardware and software verification through intermediate transformation. This approach utilizes tools like \textbf{Btor2C} combined with software verifiers, following a multi-step verification path: \(\text{BTOR2} \rightarrow \text{C} \rightarrow \text{CFA} \rightarrow \text{verification}\). The significant advantage of this workflow is its ability to leverage the extensive arsenal of software verification techniques, including sophisticated abstract interpretation, symbolic execution, and various model checking algorithms that may not be available in dedicated hardware verifiers. Nevertheless, this approach introduces an additional translation layer that can create potential semantic gaps between the original hardware design and the software representation, potentially affecting verification accuracy and performance due to the fundamental differences between hardware concurrency and software sequential execution models.
